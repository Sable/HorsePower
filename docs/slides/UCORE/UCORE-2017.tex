\documentclass[sigplan]{acmart}

% remove ACM copyright blocks (found at: http://goo.gl/6zcKds)
\settopmatter{printacmref=false}
\renewcommand\footnotetextcopyrightpermission[1]{}
\pagestyle{plain}

\begin{document}
\title{HorseIR : An Array-based Approach to SQL Queries}

\author{Hongji Chen}
\affiliation{
    \institution{McGill University}
    \city{Montreal}
    \state{Quebec}
    \country{Canada}
}
\email{hongji.chen@mail.mcgill.ca}

\maketitle

\section{Problem and Motivation}

Modern software services, such as financial analysis, operate on a vast amount
of data. To address the challenge of data querying, the design of the database
systems and query languages continuously evolve during the past decade.
Database systems, such as MonetDB\cite{MonetDBHome}, utilize the
IR-based analysis and optimization techniques to meet the performance
challenges and introduce user-defined functions (UDFs) to extend the
flexibility of the SQL standard.  However, the conventional IR design hides the
details of database querying steps (such as joining) or separates the IR with
user-defined functions. These drawbacks significantly limit the context
information propagated to the backend and render the optimizer less effective. 

In this project, we purpose an array-based IR, named HorseIR, which explores the
details of query fully and connects to user-defined functions transparently. By
replacing SQL and user-defined functions with built-in array-based primitives,
HorseIR will enable the optimizer in back-end to employ techniques developed in
programming language research field such as data-flow analysis, in the hope of
handling more flexible queries efficiently. 

\section{Background and Related Work}

MonetDB is one of the pioneers in IR based execution engine design
\cite{DBLP:journals/debu/IdreosGNMMK12}. Upon receiving a SQL query, its
front-end compiles the query into a sequence of intermediate representation
codes called MonetDB Assembly Language (MAL), with each instruction mapping to
a built-in relational algebra primitive operation. The execution kernel later
interprets the instructions step by step and generates the query result.

In 1993, Arthur Whitney launched the K programming language
\cite{KLangTutorial} , as a foundation for KDB system. After ten years of
development, K programming language evolves into Q programming language
\cite{QLangTutorial} which provides a more readable interface. Both languages
adopt the idea of array programming; hence they can be easily parallelized
using SIMD or MIMD during execution \cite{HowFastCanAPLBe}. 



\section{Approach and Uniqueness}

The state of the art architecture of MonetDB kernel inspires the design of
HorseIR. Similar to MonetDB, we introduce an intermediate representation, to
handle the complexity of SQL query interpretation.  However, unlike MonetDB,
the syntax and primitives follow the array programming style. HorseIR is a
static single assigned (SSA) statically typed array programming language. It
implements most of the primitives in array programming and database-specific
data types introduced in K and Q programming languages. Comparing with the
traditional approaches to SQL queries, HorseIR has the following advantages,

\paragraph{Array-based IR with Rich Set of Built-in Primitives}
In MonetDB, each primitive captures the operation in relational algebra.
Although this simplifies the design of the execution engine, it hides the
details from the optimizer. Thus in HorseIR, we expose such information to the
back-end by introducing an array-based IR. The primitives in HorseIR are basic
array-based arithmetic (e.g. factorial) or database-specific operations (e.g.
table loading). Thus we can replace each relational operation with one or more
array-based primitives. 

\paragraph{Programming Language Optimization for Database}
In the traditional approach, the optimizations for database queries focused
on relational algebra and execution plan level. Thus, in MonetDB and many other
database systems, minimal optimizations are applied on the intermediate
representation level. By introducing array-based HorseIR, we introduce the
techniques developed in programming language research field to the database
query processing.

\paragraph{Extensible Framework for User-defined Functions (UDF)}
There is no specification for user-defined functions in SQL standard. However,
to improve the flexibility of the SQL queries, many database systems have
introduced UDF as a language extension. However, the implementation and
optimization for UDF introduce complexity in system design. The traditional
approach UDF is either reduce the language flexibility (used in MonetDB, all
UDF are implemented using hand-written MAL \cite{MALUDF}) or require external
linkage \cite{ExternUDF}. Our solution is to compile the UDF written in an
arbitrary language into HorseIR and later compile the generated intermediate
code with the SQL queries. HorseIR allows flexible UDF implementation and
efficient code generation. Further, we can apply compiler optimization
techniques to optimize the UDF, such as function inlining.

\paragraph{Efficient Parallel Code Generation}
Recent research attempted to introduce parallel computation into database
queries using both CPU\cite{DBLP:conf/sigmod/PolychroniouRR15,
DBLP:conf/sigmod/ZhouR02} and GPU \cite{DBLP:conf/ica3pp/CremerBMM16}.
These research shows that the performance of database query is improved
significantly using the parallel physical architecture. As HorseIR is an
array-based programming language, it is easy to parallelize array-based
primitives \cite{DBLP:conf/pldi/ImamSLK14}.


\section{Results and Contributions}
\subsection*{Language Design}
Hanfeng sketches the general structure and back-end built-in primitives of
HorseIR. I later add the type system to the language. The type system allows
static type checking and function overloading. These features enhance the
performance and the flexibility of HorseIR. The type system classifies the
variables into four unique classes: primitives, dictionaries, lists, and
enumerations.  Similar to MonetDB which uses "any" keyword to declare
polymorphic user-defined functions \cite{MonetDBPolymorphism}, HorseIR system
allows polymorphic built-in and user-defined function declarations by
introducing a particular data type called wildcard type. Multiple primitive
built-in function implementations are allowed in HorseIR. The overloading
provided by the type system allows the interpreter to select the most efficient
implementation. Unlike MonetDB which defers the type checking and function
dispatching to runtime, HorseIR will resolve these during compile time, to
minimize runtime overhead and improve the performance.

\subsection*{Front-end and Interpreter Implementation}
In the summer, I have finished the implementation of the front-end of HorseIR.
The front-end provides a lexer and a parser for HorseIR. Both the lexer and the
parser are implemented using ANTLR4\cite{ANTLRHome} , which provides an
user-friendly interface generating robust LL parser. After the parsing, the
front-end will transform the parse tree into a customized abstract syntax tree.
The abstract syntax tree later serves as a foundation for program analysis and
transformation. From the abstract syntax tree, the front-end can generate the
control-flow graph and static call graph. An execution engine can later use
this information to interpret the program efficiently. The last component of
the front-end is the interpreter. Connected to Hanfeng's backend, the
interpreter traverses across the abstract syntax tree, and interpret statements
individually.

\section{Conclusion and Future Work}

In this project, we purpose a new intermediate representation on SQL database
queries. The IR provide the queries with high-performance implementation and
flexible extension. The new system would enable high-performance in-memory data
access, that is essential to many software applications. The project also
provides a new aspect in database query optimization research by employing
programming language techniques. 

By the end of the summer, I have finished the implementation of the
interpreter, and a subset of the analysis framework. In the future, I would
like to enhance the analysis framework to support more complex data-flow
analysis. Currently, all SQL queries are translated into HorseIR manually; we
will continuously work on this project to implement an automatic translation
unit which conducts this process automatically.

\section*{Acknowledgements}
This project was done as a summer research project for me. I would like to
thank Hanfeng Chen for his guidance in the array-programming and database
research field, and Prof. Laurie Hendren for her encouragement throughout the
summer. 

\bibliographystyle{ACM-Reference-Format}
\bibliography{UCORE-2017}
\end{document}
