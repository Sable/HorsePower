\subsection{return}

\begin{verbatim}
return operand
\end{verbatim}

\head{A \texttt{return} statement is well-typed if the \texttt{operand} has the same type as the return type of the enclosing function.}

\begin{align*}
\frac
{V,T,F\vdash op:\tau\ \ F=\tau\ \ V,T,F\vdash rest}
{V,T,F\vdash return\ op;rest}
\end{align*}

\subsection{goto}

\begin{verbatim}
goto [label_name]
\end{verbatim}

\head{A \texttt{goto} statement with a label is an unconditional jump.  It is well-typed if the label has a \texttt{label} type.}

\begin{align*}
\frac
{V,T,F\vdash n:label\ \ V,T,F\vdash rest}
{V,T,F\vdash goto\ [n]; rest}
\end{align*}

\begin{verbatim}
goto [label_name] label_flag
\end{verbatim}

\head{A \texttt{goto} statement with a label and a flag is a conditional jump.  It is well typed if the label has a \texttt{label} type and the flag has a \texttt{boolean} type.}

\begin{align*}
\frac
{V,T,F\vdash n:label,f:bool\ \ V,T,F\vdash rest}
{V,T,F\vdash goto\ [n]\ f; rest}
\end{align*}

\subsection{Block}

\begin{verbatim}
{
	// statements
}
\end{verbatim}

\head{It checks if all statements in the block are well-typed.  A block opens a new scope in the symbol table.}